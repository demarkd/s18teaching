\documentclass[english]{article}
\newcommand{\G}{\overline{C_{2k-1}}}
\usepackage[latin9]{inputenc}
\usepackage{amsmath}
\usepackage{amssymb}
\usepackage{lmodern}
\usepackage{mathtools}
\usepackage[inline]{enumitem}
\usepackage{multicol}
%\usepackage{natbib}
%\bibliographystyle{plainnat}
%\setcitestyle{authoryear,open={(},close={)}}content...
\let\avec=\vec
\renewcommand\vec{\mathbf}
\renewcommand{\d}[1]{\ensuremath{\operatorname{d}\!{#1}}}
\newcommand{\pydx}[2]{\frac{\partial #1}{\partial #2}}
\newcommand{\dydx}[2]{\frac{\d #1}{\d #2}}
\newcommand{\ddx}[1]{\frac{\d{}}{\d{#1}}}
\newcommand{\hk}{\hat{K}}
\newcommand{\hl}{\hat{\lambda}}
\newcommand{\ol}{\overline{\lambda}}
\newcommand{\om}{\overline{\mu}}
\newcommand{\all}{\text{all }}
\newcommand{\valph}{\vec{\alpha}}
\newcommand{\vbet}{\vec{\beta}}
\newcommand{\vT}{\vec{T}}
\newcommand{\vN}{\vec{N}}
\newcommand{\vB}{\vec{B}}
\newcommand{\vX}{\vec{X}}
\newcommand{\vx}{\vec {x}}
\newcommand{\vn}{\vec{n}}
\newcommand{\vxs}{\vec {x}^*}
\newcommand{\vV}{\vec{V}}
\newcommand{\vTa}{\vec{T}_\alpha}
\newcommand{\vNa}{\vec{N}_\alpha}
\newcommand{\vBa}{\vec{B}_\alpha}
\newcommand{\vTb}{\vec{T}_\beta}
\newcommand{\vNb}{\vec{N}_\beta}
\newcommand{\vBb}{\vec{B}_\beta}
\newcommand{\bvT}{\bar{\vT}}
\newcommand{\ka}{\kappa_\alpha}
\newcommand{\ta}{\tau_\alpha}
\newcommand{\kb}{\kappa_\beta}
\newcommand{\tb}{\tau_\beta}
\newcommand{\hth}{\hat{\theta}}
\newcommand{\evat}[3]{\left. #1\right|_{#2}^{#3}}
\newcommand{\prompt}[1]{\begin{prompt*}
		#1
\end{prompt*}}
\newcommand{\vy}{\vec{y}}
\DeclareMathOperator{\sech}{sech}
\DeclarePairedDelimiter\abs{\lvert}{\rvert}%
\DeclarePairedDelimiter\norm{\lVert}{\rVert}%
%\newcommand{\dis}[1]{\begin{align*}
%	#1
%	\end{align*}}
\newcommand{\dis}[1]{$\displaystyle{#1}$}
\newcommand{\LL}{\mathcal{L}}
\newcommand{\RR}{\mathbb{R}}
\newcommand{\NN}{\mathbb{N}}
\newcommand{\ZZ}{\mathbb{Z}}
\newcommand{\QQ}{\mathbb{Q}}
\newcommand{\Ss}{\mathcal{S}}
\newcommand{\BB}{\mathcal{B}}
\usepackage{graphicx}
\setlength{\columnseprule}{1pt}


% Swap the definition of \abs* and \norm*, so that \abs
% and \norm resizes the size \dis{\tan (\pi/6)}of the brackets, and the ake no more than 5 minutes on question 0}
%%\prob{0}\textbf{Oops! Looks like the American dollar went belly-up last week, and as a result, society has completely collapsed across the globe!\footnote{Maybe next time, consider using something less volatile as the world's reserve currency\textemdash for now though, tough luck.} Eventually, maybe civilization will rebuild, but for now, it's total chaos\textemdash there's no longer such thing as currency, half the city is on fire, and the police and army have both fractured into a complicated web of warring factions. You and your group have to band up to survive. In the space below, determine a) what your top priorities for survival should be in the immediate term and b) what role each member will play for your team.}
% starred version does not.
%\makeatletter
%\let\oldabs\abs
%\def\abs{\@ifstar{\oldabs}{\oldabs*}}
%content...
%\let\oldnorm\norm
%\def\norm{\@ifstar{\oldnorm}{\oldnorm*}}
%\makeatother
\newenvironment{subproof}[1][\proofname]{%
	\renewcommand{\qedsymbol}{$\blacksquare$}%
	\begin{proof}[#1]%
	}{%
	\end{proof}%
}

\usepackage{centernot}
\usepackage{dirtytalk}
\usepackage{calc}
\newcommand{\prob}[1]{\setcounter{section}{#1-1}\section{}}
\newcommand{\prt}[1]{\setcounter{subsection}{#1-1}\subsection{}}
\newcommand{\pprt}[1]{{\textit{{#1}.)}}\newline}
\renewcommand\thesubsection{\alph{subsection}}
\usepackage[sl,bf,compact]{titlesec}
\titlelabel{\thetitle.)\quad}
\DeclarePairedDelimiter\floor{\lfloor}{\rfloor}
\makeatletter

%\newcommand*\pFqskip{8mu}
%\catcode`,\activecontent...
%\newcommand*\pFq{\begingroup
%	\catcode`\,\active
%	\def ,{\mskip\pFqskip\relax}%
%	\dopFq
%}
%\catcode`\,12
%\def\dopFq#1#2#3#4#5{%
%	{}_{#1}F_{#2}\biggl(\genf\dis{\tan (\pi/6)}rac..{0pt}{}{#3}{#4}|#5\biggr
%	)%
%	\endgroup
%}
\def\res{\mathop{Res}\limits}
% Symbols \wedge and \vee from mathabx
% \DeclareFontFamily{U}{matha}{\hyphenchar\font45}
% \DeclareFontShape{U}{matha}{m}{n}{content...
%       <5> <6> <7> <8> <9> <\dis{\tan (\pi/6)}10> gen * matha
%       <10.95> matha10 <12> <14.4> <17.28> <20.74> <24.88> matha12
%       }{}
% \DeclareSymbolFont{matha}{U}{matha}{m}{n}ake no more than 5 minutes on question 0}
%%\prob{0}\textbf{Oops! Looks like the American dollar went belly-up last week, and as a result, society has completely collapsed across the globe!\footnote{Maybe next time, consider using something less volatile as the world's reserve currency\textemdash for now though, tough luck.} Eventually, maybe civilization will rebuild, but for now, it's total chaos\textemdash there's no longer such thing as currency, half the city is on fire, and the police and army have both fractured into a complicated web of warring factions. You and your group have to band up to survive. In the space below, determine a) what your top priorities for survival should be in the immediate term and b) what role each member will play for your team.}
% \DeclareMathSymbol{\wedge}         {2}{matha}{"5E}
% \DeclareMathSymbol{\vee}           {2}{matha}{"5F}content...
% \makeatother

%\titlelabel{(\thesubsection)}
%\titlelabel{(\thesubsection)\quad}
\usepackage{listings}
\lstloadlanguages{[5.2]Mathematica}
\usepackage{babel}
\newcommand{\ffac}[2]{{(#1)}^{\underline{#2}}}
\usepackage{color}
\usepackage{amsthm}
\newtheorem{theorem}{Theorem}[section]
%\newtheorem*{theorem*}{Theorem}[section]
\newtheorem{conj}[theorem]{Conjecture}
\newtheorem{corollary}[theorem]{Corollary}
\newtheorem{example}[theorem]{Example}
\newtheorem{lemma}[theorem]{Lemma}
\newtheorem*{lemma*}{Lemma}
\newtheorem{problem}[theorem]{Problem}
\newtheorem{proposition}[theorem]{Proposition}
\newtheorem*{prop*}{Proposition}
\newtheorem*{corollary*}{Corollary}
\newtheorem{fact}[theorem]{Fact}

\newtheorem*{claim*}{Claim}
\newcommand{\claim}[1]{\begin{claim*} #1\end{claim*}}
%organizing theorem environments by style--by the way, should we really have definitions (and notations I guess) in proposition style? it makes SO much of our text italicized, which is weird.
\theoremstyle{remark}
\newtheorem{remark}{Remark}[section]

\theoremstyle{definition}
\newtheorem{definition}[theorem]{Definition}
\newtheorem{notation}[theorem]{Notation}
\newtheorem*{notation*}{Notation}

%FINAL
\newcommand{\due}{30 January 2018} 
\RequirePackage{geometry}
\geometry{margin=.7in}
\usepackage{todonotes}
\title{Worksheet for 1/30/18}
\author{David DeMark}
\date{\due}
\usepackage{fancyhdr}
\pagestyle{fancy}
\fancyhf{}
\rhead{Groupwork 1/30/18}
\chead{\due}
\lhead{MATH 1271-052\&053}
\cfoot{\thepage}
% %%
%%
%%
%DRAFT

%\usepackage[left=1cm,right=4.5cm,top=2cm,bottom=1.5cm,marginparwidth=4cm]{geometry}
%\usepackage{todonotes}inparaenum
% \title{MATH 8669 Homework 4-DRAFT}
% \usepackage{fancyhdr}
% \pagestyle{fancy}
% \fancyhf{}\dis{\tan (\pi/6)}
% \rhead{David DeMark}
% \lhead{MATH 8669-Homework 4-DRAFT}
% \cfoot{\thepage}

%PROBLEM SPEFICIC

\newcommand{\lint}{\underline{\int}}
\newcommand{\uint}{\overline{\int}}
\newcommand{\hfi}{\hat{f}^{-1}}
\newcommand{\tfi}{\tilde{f}^{-1}}
\newcommand{\tsi}{\tilde{f}^{-1}}
\newcommand{\PP}{\mathcal{P}}
\newcommand{\nin}{\centernot\in}
\newcommand{\seq}[1]{({#1}_n)_{n\geq 1}}
\usepackage{array}
\newcolumntype{M}[1]{>{\centering\arraybackslash}m{#1}}
\newcolumntype{N}{@{}m{0pt}@{}}
\input ArtNouvc.fd
\newcommand*\initfamily{\usefont{U}{ArtNouvc}{xl}{n}}
\newcommand{\ansl}{~\underline{\hspace{1.5cm}}}
\usepackage{pgfplots}
\pgfplotsset{my style/.append style={axis x line=middle, axis y line=
		middle, xlabel={$x$}, ylabel={$y$}, axis equal,width=\textwidth }}
\pgfplotsset{every x tick label/.append style={font=\small, yshift=0.5ex}}
\pgfplotsset{every y tick label/.append style={font=\small, xshift=0.5ex}}
%\usepackage{paralist}
\begin{document}
\prob{1} For each of the functions listed, \textbf{(i)} identify any points of discontinuity and \textbf{(ii)} say whether it is continuous from the left, right, or neither.

\prt{1} $$f(x)=\frac{x+4}{2x^2+7x-4}$$	
\begin{proof}[Response]
	In Stewart, \S2.5, theorem 4, it is stated that if $g(x)$ and $h(x)$ are continuous at $a$, then $\frac{g(x)}{h(x)}$ is continuous at $a$ \textit{unless} $h(a)=0$. Polynomials are continuous everywhere, so we need only find the zeros of $2x^2+7x-4$. By factoring $2x^2+7x-4=(2x-1)(x+4)$, we see that $f$ is continuous everywhere except $x=\frac{1}{2},-4$. As $f(\frac{1}{2})$ and $f(4)$ are undefined, $f$ cannot possible be continuous (from the right or from the left, for that matter) at either point as the definition of continuity ($f(x)$ is continuous at $a$ $\iff$ $\lim_{x\to a}f(x)=f(a)$) requires $f(a)$ to be defined. 
\end{proof}\vspace{1cm}
\prt{2}$$f(x)=\frac{x^2-1}{x^2+1}$$
\begin{proof}[Response]
	$x^2+1$ is never 0 for any real $x$. Thus, $f(x)$ is continuous everywhere!
\end{proof}\vspace{1cm}
\prt{3}
	\begin{equation*}f(x)=\begin{cases}
	\sqrt{1-x}& x\leq 1\\
	\frac{2x^3-2}{x-2}& 1<x\leq 3\\
	52\sin(\pi x)& 3<x
	
	\end{cases}\end{equation*}
\begin{proof}[Response]
 There are two steps to checking for discontinuities here: check for any discontinuities within each \say{leg} of the piecewise, then check for discontinuities at the \say{seams} (the points at which we change from one function to another). $\sqrt{1-x}$ is continuous everywhere it is defined except at $x=1$ where it is continuous only from the left\textemdash however, we shouldn't take that to mean that $f$ has a discontinuity at $1$, as 1 is one of the \say{seams} of the piecewise, and we'll handle it later. $\frac{2x^3-2}{x-2}$ is undefined at $x=2$, which is in the domain of the second leg of the piecewise, so we have a discontinuity (not continuous from right \textit{or} left) at $x=2$. Finally, $52\sin(\pi x)$ is continuous everywhere. What is left to check is the two \say{seams} at $x=1$ and $x=3$. $$\lim_{x\to 1^-}f(x)=\lim_{x\to 1^-}\sqrt{1-x}=0=f(1),$$ so $f$ is at least continuous from the left there.  $$\lim_{x\to 1^+}f(x)=\lim_{x\to 1^+}\frac{2x^3-2}{x-2}=0=f(1)$$ so it is continuous from the right as well. Thus, $f$ is continuous at 1. On the other hand, $$\lim_{x\to 3^-}f(x)=\lim_{x\to 3^-}\frac{2x^3-2}{x-2}=52=f(3)$$ so $f$ is continuous from the left at 3, but $$\lim_{x\to 3^+}f(x)=\lim_{x\to 3^+}52\sin (\pi x)=52\sin (3\pi)=0\neq f(3),$$ so $f$ is \textit{not} continuous from the right at 3.
 
 To summarize, we found two discontinuities: $f$ is not continuous from either direction at $x=2$, and $f$ is continuous only from the left at $x=3$. 
\end{proof}	

%	\newpage
%\prob{2}
%Sketch a graph $y=f(x)$ for a function $f$ satisfying \textit{all} of the following properties (you do not need to provide a formula for $f$ unless you want to):\\
%\prt{1}$\displaystyle\lim_{x\to-\infty}f(x)=0$,~ $\displaystyle\lim_{x\to-2^+}f(x)=3$,~ $\displaystyle\lim_{x\to -2^{-}}f(x)=-3$, $\lim_{x\to 3}f(x)=-\infty$,~ $f$ is continuous \textit{from the right} at $x=-3$
%
%\vspace{2cm}
%
%\begin{tikzpicture}
%\begin{axis}[my style, xtick={-7,...,7}, ytick={-7,...,7},
%xmin=-8, xmax=8,  ymin=-8, ymax=8]
%\end{axis}
%\end{tikzpicture}
%\newpage\prt{2}
%$\displaystyle\lim_{x\to-\infty}f(x)=-3$,~ $\displaystyle\lim_{x\to -1^+}f(x)=-2$,~ $\displaystyle\lim_{x\to 0}f(x)=-\infty$,~ $\displaystyle\lim_{x\to 2}f(x)=3$,~ $f(2)=6$,\newline $\displaystyle\lim_{x\to \infty} f(x)=2$,~ there are only two real numbers at which $f$ is \textit{not} continuous.\\
%
%\vspace{2cm}
%\begin{tikzpicture}
%\begin{axis}[my style, xtick={-7,...,7}, ytick={-7,...,7},
%xmin=-8, xmax=8,  ymin=-8, ymax=8]
%\end{axis}
%\end{tikzpicture}
%\newpage
%\prob{3} Evaluate the following limits.\begin{multicols}{2}
%
%
%\prt{1} $$\lim_{x\to 2}\frac{1}{x-2}$$
%\begin{proof}[Response]
%	$\displaystyle\lim_{x\to 2^-}\frac{1}{x-2}=-\infty$, $\displaystyle\lim_{x\to 2^+}\frac{1}{x-2}=\infty$, so $\displaystyle\lim_{x\to 2}\frac{1}{x-2}$ does not exist.
%\end{proof}
%
%\prt{2} $$\lim_{x\to 2}\frac{1}{(x-2)^2}$$
%\begin{proof}[Response]
%		$\displaystyle\lim_{x\to 2^-}\frac{1}{(x-2)^2}=\infty$, $\displaystyle\lim_{x\to 2^+}\frac{1}{(x-2)^2}=\infty$, so $\displaystyle\lim_{x\to 2}\frac{1}{(x-2)^2}=\infty$
%\end{proof}
%\prt{3}
%$$\lim_{x\to \infty}\frac{2+5x^2}{1+x-x^2}$$
%\begin{proof}[Response]
%	Multiply through by $\displaystyle\frac{1/x^2}{1/x^2}=1$ to get $\displaystyle\lim_{x\to \infty}\frac{\frac{2}{x^2}+5}{\frac{1}{x^2}+\frac{1}{x}-1}=-5$ as each $c/x^r$ term goes to $0$. 
%\end{proof} \columnbreak
%\prt{4}
%$$\lim_{x\to -\infty} \frac{1-x^6}{1+x^5}$$\begin{proof}[Response]
%	Multiply through by $\displaystyle\frac{1/x^5}{1/x^5}=1$ to get $\displaystyle\lim_{x\to -\infty}\frac{\frac{1}{x^5}-x}{\frac{1}{x^5}+1}$. Note $\displaystyle\lim_{x\to -\infty}(\frac{1}{x^5}-x)=\infty$, while $\displaystyle\lim_{x\to -\infty}\frac{1}{x^5}+1=1$. Thus, $\displaystyle\lim_{x\to -\infty}\frac{1-x^6}{1+x^5}=\infty$. 
%\end{proof}
%\prt{5}
%$$\lim_{x\to \infty}\frac{\sqrt{1+4x^6}}{2-x^3}$$\begin{proof}[Response]
%	Multiply through by $1=\frac{1/\sqrt{x^6}}{1/x^3}$ to get $\displaystyle\lim_{x\to \infty} \frac{\sqrt{\frac{1}{x^6}+4}}{\frac{2}{x^3}-1}$. Note that $\displaystyle\lim_{x\to \infty}\sqrt{\frac{1}{x^6}+4}=\sqrt{\lim_{x\to \infty}\frac{1}{x^6}+4}=\sqrt{4}=2$, while $\displaystyle\lim_{x\to \infty}\frac{2}{x^3}-1=-1$. Thus, $\displaystyle\lim_{x\to \infty}\frac{\sqrt{1+4x^6}}{2-x^3}=-2$
%\end{proof}
%\prt{6} $$\lim_{x\to \infty} e^{-x}\sin^2(x^2)$$\begin{proof}[Response]
%	Squeeze theorem! For any $x$, it is the case that $0\leq \sin^2(x^2)\leq 1$, and $e^{-x}>0$ for any $x$, so we can multiply through and now now have that $0=0*e^{-x}\leq e^{-x}\sin^2(x^2)\leq e^{-x}$. The squeeze theorem now says $\displaystyle\lim_{x\to\infty}0\leq\lim_{x\to\infty} e^{-x}\sin^2(x^2)\leq\lim_{x\to\infty} e^{-x}$. Those two outer limits are $0$, so we must have $\displaystyle\lim_{x\to \infty}e^{-x}\sin^2(x^2)=0$. 
%\end{proof}
%\end{multicols}
%

\end{document}
