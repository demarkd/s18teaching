\documentclass[english]{article}
\newcommand{\G}{\overline{C_{2k-1}}}
\usepackage[latin9]{inputenc}
\usepackage{amsmath}
\usepackage{amssymb}
\usepackage{lmodern}
\usepackage{mathtools}
\usepackage[inline]{enumitem}
\usepackage{multicol}
%\usepackage{natbib}
%\bibliographystyle{plainnat}
%\setcitestyle{authoryear,open={(},close={)}}content...
\let\avec=\vec
\renewcommand\vec{\mathbf}
\renewcommand{\d}[1]{\ensuremath{\operatorname{d}\!{#1}}}
\newcommand{\pydx}[2]{\frac{\partial #1}{\partial #2}}
\newcommand{\dydx}[2]{\frac{\d #1}{\d #2}}
\newcommand{\ddx}[1]{\frac{\d{}}{\d{#1}}}
\newcommand{\hk}{\hat{K}}
\newcommand{\hl}{\hat{\lambda}}
\newcommand{\ol}{\overline{\lambda}}
\newcommand{\om}{\overline{\mu}}
\newcommand{\all}{\text{all }}
\newcommand{\valph}{\vec{\alpha}}
\newcommand{\vbet}{\vec{\beta}}
\newcommand{\vT}{\vec{T}}
\newcommand{\vN}{\vec{N}}
\newcommand{\vB}{\vec{B}}
\newcommand{\vX}{\vec{X}}
\newcommand{\vx}{\vec {x}}
\newcommand{\vn}{\vec{n}}
\newcommand{\vxs}{\vec {x}^*}
\newcommand{\vV}{\vec{V}}
\newcommand{\vTa}{\vec{T}_\alpha}
\newcommand{\vNa}{\vec{N}_\alpha}
\newcommand{\vBa}{\vec{B}_\alpha}
\newcommand{\vTb}{\vec{T}_\beta}
\newcommand{\vNb}{\vec{N}_\beta}
\newcommand{\vBb}{\vec{B}_\beta}
\newcommand{\bvT}{\bar{\vT}}
\newcommand{\ka}{\kappa_\alpha}
\newcommand{\ta}{\tau_\alpha}
\newcommand{\kb}{\kappa_\beta}
\newcommand{\tb}{\tau_\beta}
\newcommand{\hth}{\hat{\theta}}
\newcommand{\evat}[3]{\left. #1\right|_{#2}^{#3}}
\newcommand{\prompt}[1]{\begin{prompt*}
		#1
\end{prompt*}}
\newcommand{\vy}{\vec{y}}
\DeclareMathOperator{\sech}{sech}
\DeclarePairedDelimiter\abs{\lvert}{\rvert}%
\DeclarePairedDelimiter\norm{\lVert}{\rVert}%
%\newcommand{\dis}[1]{\begin{align*}
%	#1
%	\end{align*}}
\newcommand{\dis}[1]{$\displaystyle{#1}$}
\newcommand{\LL}{\mathcal{L}}
\newcommand{\RR}{\mathbb{R}}
\newcommand{\NN}{\mathbb{N}}
\newcommand{\ZZ}{\mathbb{Z}}
\newcommand{\QQ}{\mathbb{Q}}
\newcommand{\Ss}{\mathcal{S}}
\newcommand{\BB}{\mathcal{B}}
\usepackage{graphicx}
% Swap the definition of \abs* and \norm*, so that \abs
% and \norm resizes the size \dis{\tan (\pi/6)}of the brackets, and the ake no more than 5 minutes on question 0}
%%\prob{0}\textbf{Oops! Looks like the American dollar went belly-up last week, and as a result, society has completely collapsed across the globe!\footnote{Maybe next time, consider using something less volatile as the world's reserve currency\textemdash for now though, tough luck.} Eventually, maybe civilization will rebuild, but for now, it's total chaos\textemdash there's no longer such thing as currency, half the city is on fire, and the police and army have both fractured into a complicated web of warring factions. You and your group have to band up to survive. In the space below, determine a) what your top priorities for survival should be in the immediate term and b) what role each member will play for your team.}
% starred version does not.
%\makeatletter
%\let\oldabs\abs
%\def\abs{\@ifstar{\oldabs}{\oldabs*}}
%content...
%\let\oldnorm\norm
%\def\norm{\@ifstar{\oldnorm}{\oldnorm*}}
%\makeatother
\newenvironment{subproof}[1][\proofname]{%
	\renewcommand{\qedsymbol}{$\blacksquare$}%
	\begin{proof}[#1]%
	}{%
	\end{proof}%
}

\usepackage{centernot}
\usepackage{dirtytalk}
\usepackage{calc}
\newcommand{\prob}[1]{\setcounter{section}{#1-1}\section{}}
\newcommand{\prt}[1]{\setcounter{subsection}{#1-1}\subsection{}}
\newcommand{\pprt}[1]{{\textit{{#1}.)}}\newline}
\renewcommand\thesubsection{\alph{subsection}}
\usepackage[sl,bf,compact]{titlesec}
\titlelabel{\thetitle.)\quad}
\DeclarePairedDelimiter\floor{\lfloor}{\rfloor}
\makeatletter

%\newcommand*\pFqskip{8mu}
%\catcode`,\activecontent...
%\newcommand*\pFq{\begingroup
%	\catcode`\,\active
%	\def ,{\mskip\pFqskip\relax}%
%	\dopFq
%}
%\catcode`\,12
%\def\dopFq#1#2#3#4#5{%
%	{}_{#1}F_{#2}\biggl(\genf\dis{\tan (\pi/6)}rac..{0pt}{}{#3}{#4}|#5\biggr
%	)%
%	\endgroup
%}
\def\res{\mathop{Res}\limits}
% Symbols \wedge and \vee from mathabx
% \DeclareFontFamily{U}{matha}{\hyphenchar\font45}
% \DeclareFontShape{U}{matha}{m}{n}{content...
%       <5> <6> <7> <8> <9> <\dis{\tan (\pi/6)}10> gen * matha
%       <10.95> matha10 <12> <14.4> <17.28> <20.74> <24.88> matha12
%       }{}
% \DeclareSymbolFont{matha}{U}{matha}{m}{n}ake no more than 5 minutes on question 0}
%%\prob{0}\textbf{Oops! Looks like the American dollar went belly-up last week, and as a result, society has completely collapsed across the globe!\footnote{Maybe next time, consider using something less volatile as the world's reserve currency\textemdash for now though, tough luck.} Eventually, maybe civilization will rebuild, but for now, it's total chaos\textemdash there's no longer such thing as currency, half the city is on fire, and the police and army have both fractured into a complicated web of warring factions. You and your group have to band up to survive. In the space below, determine a) what your top priorities for survival should be in the immediate term and b) what role each member will play for your team.}
% \DeclareMathSymbol{\wedge}         {2}{matha}{"5E}
% \DeclareMathSymbol{\vee}           {2}{matha}{"5F}content...
% \makeatother

%\titlelabel{(\thesubsection)}
%\titlelabel{(\thesubsection)\quad}
\usepackage{listings}
\lstloadlanguages{[5.2]Mathematica}
\usepackage{babel}
\newcommand{\ffac}[2]{{(#1)}^{\underline{#2}}}
\usepackage{color}
\usepackage{amsthm}
\newtheorem{theorem}{Theorem}[section]
%\newtheorem*{theorem*}{Theorem}[section]
\newtheorem{conj}[theorem]{Conjecture}
\newtheorem{corollary}[theorem]{Corollary}
\newtheorem{example}[theorem]{Example}
\newtheorem{lemma}[theorem]{Lemma}
\newtheorem*{lemma*}{Lemma}
\newtheorem{problem}[theorem]{Problem}
\newtheorem{proposition}[theorem]{Proposition}
\newtheorem*{prop*}{Proposition}
\newtheorem*{corollary*}{Corollary}
\newtheorem{fact}[theorem]{Fact}

\newtheorem*{claim*}{Claim}
\newcommand{\claim}[1]{\begin{claim*} #1\end{claim*}}
%organizing theorem environments by style--by the way, should we really have definitions (and notations I guess) in proposition style? it makes SO much of our text italicized, which is weird.
\theoremstyle{remark}
\newtheorem{remark}{Remark}[section]

\theoremstyle{definition}
\newtheorem{definition}[theorem]{Definition}
\newtheorem{notation}[theorem]{Notation}
\newtheorem*{notation*}{Notation}

%FINAL
\newcommand{\due}{3 October 2017} 
\RequirePackage{geometry}
\geometry{margin=.7in}
\usepackage{todonotes}
\title{Worksheet for 10/3\textemdash David's Solutions}
\author{David DeMark}
\date{\due}
\usepackage{fancyhdr}
\pagestyle{fancy}
\fancyhf{}
\rhead{Groupwork 10/3/17}
\chead{\due}
\lhead{MATH 1271-012\&016}
\cfoot{\thepage}
% %%
%%
%%
%DRAFT

%\usepackage[left=1cm,right=4.5cm,top=2cm,bottom=1.5cm,marginparwidth=4cm]{geometry}
%\usepackage{todonotes}inparaenum
% \title{MATH 8669 Homework 4-DRAFT}
% \usepackage{fancyhdr}
% \pagestyle{fancy}
% \fancyhf{}\dis{\tan (\pi/6)}
% \rhead{David DeMark}
% \lhead{MATH 8669-Homework 4-DRAFT}
% \cfoot{\thepage}

%PROBLEM SPEFICIC

\newcommand{\lint}{\underline{\int}}
\newcommand{\uint}{\overline{\int}}
\newcommand{\hfi}{\hat{f}^{-1}}
\newcommand{\tfi}{\tilde{f}^{-1}}
\newcommand{\tsi}{\tilde{f}^{-1}}
\newcommand{\PP}{\mathcal{P}}
\newcommand{\nin}{\centernot\in}
\newcommand{\seq}[1]{({#1}_n)_{n\geq 1}}
\usepackage{array}
\newcolumntype{M}[1]{>{\centering\arraybackslash}m{#1}}
\newcolumntype{N}{@{}m{0pt}@{}}
\input ArtNouvc.fd
\newcommand*\initfamily{\usefont{U}{ArtNouvc}{xl}{n}}
\newcommand{\ansl}{~\underline{\hspace{1.5cm}}}
%\usepackage{paralist}
\begin{document}
%{\bf \Large Take no more than 5 minutes on question 0\textellipsis}
%\prob{0}\textbf{Oops! Looks like the American dollar went belly-up last week, and as a result, society has completely collapsed across the globe!\footnote{Maybe next time, consider using something less volatile as the world's reserve currency\textemdash for now though, tough luck.} Eventually, maybe civilization will rebuild, but for now, it's total chaos\textemdash there's no longer such thing as currency, half the city is on fire, and the police and army have both fractured into a complicated web of warring factions. You and your group have to band up to survive. In the space below, determine a) what your top priorities for survival should be in the immediate term and b) what role each member will play for your team.}\vspace{4cm}

		\prob{1}{(Pre-calc review) Each of the following prompts corresponds to exactly one of the options from the \say{answer bank.} Each answer in the answer bank corresponds to at most one prompt. Match answers to prompts! \textbf{!!!DO NOT USE A CALCULATOR OR LOOK UP ANY FORMULAS BESIDES THE ONES GIVEN!!!}}\newline
		\subsection*{Formulas \& Definitions}
		
		$\tan(x)=\frac{\sin(x)}{\cos(x)}$,
		$	\sec(x)=\frac{1}{\cos(x)}$,
		$	\csc(x)=\frac{1}{\sin(x)}$,
		$	\cot(x)=\frac{1}{\tan(x)}$,
$1=\cos^2x+\sin^2x$

		\subsection*{Round 1}
		\begin{enumerate}
			\item An equation giving a circle of radius 2 centered at $(2,3)$: \textbf{E}
			\item An equation giving a line perpendicular to the line between $(1,0)$ and $(5,3)$:\textbf{A}
			\item \dis{\tan (\pi/6)-\cot (\pi/6)}:\textbf{I}
			\item An equation giving a line which intersects the parabola $y=x^2$ \textit{exactly} once: \textbf{N/A}
			\item \dis{\frac{\ln32+\ln2}{\ln4}}:\textbf{D}
			\item An equation giving a circle of radius 4 centered at $(-2,-3)$:\textbf{K}
			\item A solution to the equation $y=x^2+\frac{2}{3}x-\frac{53}{6}$: \textbf{N/A} (should be $y=\frac{4}{3}x-\frac{4}{9}$)
			\item \dis{\frac{\sin{\pi/2}}{\cos{\pi/3}}}: \textbf{H}
		\end{enumerate}
		\textit{\textbf{Answer Bank:}}
		\fbox{\begin{minipage}{40em}
				\begin{enumerate*}[label=\hspace{.35cm}\textbf{\textit{\Alph*)\hspace{.05cm}}}]
					\item \dis{y=-\frac{4}{3} x+\frac{6}{5}} \item \dis{(x-2)^2+(y-3)^2=16}\item \dis{y=-\frac{3}{4}x+\frac{6}{8}} \item 3\newline\item \dis{(x-2)^2+(y-3)^2=4} \item \dis{\frac{\sqrt{3}}{3}}\item \dis{(x+2)^2+(y+3)^2=2}\item 2 \item \dis{\frac{-2}{\sqrt{3}}} \item \dis{y=\frac{4}{3}x-\frac{1}{9}} \item \dis{(x+2)^2+(y+3)^2=16} \item\dis{ \frac{3}{\sqrt{3}}} \item \dis{\sqrt{6}-\frac{1}{3}}
				\end{enumerate*}
		\end{minipage}}
	\subsection*{Explanations}
	\begin{enumerate}
	\item The equation for a circle centered at $(x_0,y_0)$ of radius $r$ is given by an equation involving the \emph{distance formula}:$$(y-y_0)^2+(x-x_0)^2=r^2.$$ Plugging in $x_0=2,y_0=3,r=2$, we get $$(y-3)^2+(x-2)^2=4$$ as in E).
	\item Note the line between $(1,0)$ and $(5,3)$ is of slope $\frac{3}{4}$. For a line of slope $m$, the slope of any line perpendicular to it is $\frac{-1}{m}$. Thus, the slope of the line we're looking for is $\frac{-4}{3}$. A is the only answer which gives a line of that slope.
	\item Use special right triangles (in particular the 30-60-90 triangle) to compute $\tan(\pi/6)=\frac{1}{\sqrt{3}}$. Then, $\tan(\pi/6)-\cot(\pi/6)=\frac{1}{\sqrt{3}}-\sqrt{3}$. Finding a common denominator, we have $ \tan(\pi/6)-\cot(\pi/6)=\frac{1-3}{\sqrt{3}}=-\frac{2}{\sqrt{3}}$.\
\item This one can only really be done by testing any lines we haven't eliminated and solving for an intersection point. If we try solving for intersections between the line in C and $y=x^2$, we need to solve the equation $x^2=-\frac{3}{4}x+\frac{6}{8}$, or $x^2+\frac{3}{4}x-\frac{6}{8}=0$. The quadratic formula gives $$x=\frac{1}{8} \left(-3\pm\sqrt{57}\right),$$ which, in particular, is two different $x$-coordinates (we could have also seen this by noting $b^2-4ac\neq 0$ for the equation $x^2+bx+c=0$). Unfortunately the same happens with the equation in J) because I screwed up making this (oops, sorry.). However, replacing it with what it \textit{should} be, solving $x^2=\frac{4}{3}x-\frac{4}{9}$ gives only one solution, $\frac{2}{3}$. 
\item
Start with $x=\frac{\ln32+\ln2}{\ln4}$. Then, $x\ln4=\ln32+\ln2$. Exponentiate each side with base $e$. Now, $e^{x\ln 4}=e^{\ln32+\ln2}$. Recall that by definition of natural logarithm, for any $a$, $e^{\ln a}=\ln(e^a)=a$. Two other useful facts are that (for any $z,a,b$) $z^{a*b}=(z^a)^b$ and $z^{a+b}=z^az^b$. Now, we can use these to first rewrite $e^{x\ln4}$ as $(e^{\ln4})^x=4^x$, and $e^{\ln 32+\ln2}=e^{\ln 32}e^{\ln2}=32*2=64$. Now, our equation from above is $4^x=64$. This is solved by $x=3$.

\item See explanation 1.
\item oops
\item Use the unit circle and/or special right triangles to find $\sin(\pi/2)=1$ and $\cos(\pi/3)=1/2$.
	\end{enumerate}
	\newpage
		\subsection*{Round 2}
		
		\textcolor{red}{As it turns out, there are more mistakes here then I realized, so this is pretty messy. My apologies, this is sloppy work on my part.}
	\begin{enumerate}
		\item $\abs{\tan x}$: \textbf{N/A} (but ALMOST D)
		\item \dis{\frac{1-2\cos^2x}{\cos^2x-\sin^2x}}: \textbf{-1}
		\item \dis{\frac{\cos^2(x)-\sin^2(x)}{2\cos^2(x)-1}}: \textbf{C}
		\item $\abs{\cot x}$: \textbf{E}
		\item \dis{\frac{-\cos^2x}{\sin x-1}}: \textbf{F}
		\item $\cot x-\sin x + \sin^3 x$: \textbf{N/A}
		\item $\abs{\csc x}$: \textbf{B}
	\end{enumerate}
	\textit{\textbf{Answer Bank:}}
	\fbox{\begin{minipage}{40em}
			\begin{enumerate*}[label=\hspace{.35cm}\textbf{\textit{\Alph*)\hspace{.05cm}}}]
			\item \dis{\frac{1-\cos x}{\sin x\sec x}}\item\dis{ \sqrt{\frac{1+\tan^2x}{\tan^2x}}}	\item 1 \item \dis{\sqrt{1-\sec^2(x)}} \item \dis{\frac{\sqrt{1-\sin^2(x)}}{\sqrt{1-\cos^2(x)}}}\newline \item $\sin x+1$
			\end{enumerate*}
	\end{minipage}}
\subsection*{Explanations}
\begin{enumerate}
\item D should have actually been $\sqrt{\sec^2x-1}$. If it were that, we would get that \begin{align*}
	\sec^2x-1&=\frac{1}{\cos^2x}-1\\&=\frac{1-\cos^2x}{\cos^2x}\\&=\frac{\sin^2x}{\cos^2x}\\&=\tan^2x.
\end{align*}
\item The identity on the first page implies $\cos^2x+\sin^2x-1=0$. We may apply this freely. We add $0=\cos^2x+\sin^2x-1$ to the numerator to yield $1-2\cos^2x=\sin^2x-\cos^2x$.
\item Note that the expression for \#3 is simply the negative of the reciprocal of the expression in \#2!!
\item $1-\sin^2x=\cos^2x$ and $1-\cos^2x=\sin^2x$, so the expression for E) can be written $\sqrt{\frac{\cos^2x}{\sin^2x}}=\sqrt{\cot^2x}=\abs{\cot x}$.
\item $-\cos^2x=\sin^2x-1=(\sin x-1)(\sin x+1)$. Now, dividing by $\sin x-1$ cancels one of those to yield $\sin x+1$. 
\item \textcolor{red}{Oy vey.} Yeah, I screwed this one up. Start with the expression in A and interpret $1/\sec x=\cos x$:
\begin{align*}
	\frac{1-\cos x}{\sin x \sec x}&= \frac{\cos x-\cos^2x}{\sin x}\\&=\cot x-\frac{\cos^2x}{\sin x}
\end{align*}
Now, use $-\cos^2x=\sin^2x-1$. We yield $\cot x-\frac{\sin^2x-1}{\sin x}$. Unfortunately, rather than dividing by $\sin$, I multiplied to get the monstrosity above. Item 6 should have been $\cot x-\sin x+\csc x$. 
\item Note $\frac{1+\tan^2x}{\tan^2x}=1+\cot^2x=1+\frac{\cos^2x}{\sin^2x}$. Now, find a common denominator and yield $\frac{\sin^2x+\cos^2x}{\sin^2x}=\frac{1}{\sin^2x}=\csc^2x$. 
\end{enumerate}
\vspace{5cm}
\prob{2}\prt{1}
Let $x_0=1$, and $f(x)=x^2-2x$. Compute the slope of the secant line between $x_0$ and $x_1=5$, $x_2=2$, $x_3=1.5$, $x_4=1.1$ and $x_5=1.01$. Use this to find an estimate for the slope of the tangent line to $y=f(x)$ at $x_0$.

\begin{proof}[Response]
	The following gives a table of secant-slopes using the formula $\text{Slope}=\frac{f(x)-f(1)}{x-1}$:
	\begin{table}[h!]\begin{tabular}{c|c}
			$x$&\text{Slope}\\\hline
		5 & 4 \\
		2 & 1 \\
		1.5 & 0.5 \\
		1.1 & 0.1 \\
		1.01 & 0.01 \\\end{tabular}\end{table}

The slope appears to be approaching 0.\end{proof}
	\vspace{4cm}
\prt{2} Use the same techniques to estimate the slope of the tangent line for $f(x)=x^3$ at $x_0=1$.
\begin{proof}[Response]
	We use the same technique:\begin{table}[h!]\begin{tabular}{c|c}
			$x$&\text{Slope}\\\hline
		5 & 31 \\
		2 & 7 \\
		1.5 & 4.75 \\
		1.1 & 3.31 \\
		1.01 & 3.0301 \\
		1.001 & 3.003 \\\end{tabular}\end{table}
	
	The slope appears to be converging towards 3.

\end{proof}

\end{document}
