% Exam Template for UMTYMP and Math Department courses
%
% Using Philip Hirschhorn's exam.cls: http://www-math.mit.edu/~psh/#ExamCls
%
% run pdflatex on a finished exam at least three times to do the grading table on front page.
%
%%%%%%%%%%%%%%%%%%%%%%%%%%%%%%%%%%%%%%%%%%%%%%%%%%%%%%%%%%%%%%%%%%%%%%%%%%%%%%%%%%%%%%%%%%%%%%

% These lines can probably stay unchanged, although you can remove the last
% two packages if you're not making pictures with tikz.
\documentclass[11pt,fleqn]{exam}
\RequirePackage{amssymb, amsfonts, amsmath,amsthm, mathtools, latexsym, verbatim, xspace, setspace}
\RequirePackage{tikz, pgflibraryplotmarks}

% By default LaTeX uses large margins.  This doesn't work well on exams; problems
% end up in the "middle" of the page, reducing the amount of space for students
% to work on them.
\usepackage[margin=1in]{geometry}


% Here's where you edit the Class, Exam, Date, etc.
\newcommand{\class}{Math 1271 - Lecture 050 }
\newcommand{\term}{Spring 2018}
\newcommand{\examnum}{Quiz VI}
\newcommand{\examdate}{3/8/18}
\newcommand{\timelimit}{20 Minutes}
\DeclarePairedDelimiter\abs{\lvert}{\rvert}%
% For an exam, single spacing is most appropriate
\singlespacing
% \onehalfspacing
% \doublespacing

% For an exam, we generally want to turn off paragraph indentation
\parindent 0ex
\def\changemargin#1#2{\list{}{\rightmargin#2\leftmargin#1}\item[]}
\let\endchangemargin=\endlist 
%%\usepackage{pgfplots}
%%\pgfplotsset{my style/.append style={axis x line=middle, axis y line=
%		middle, xlabel={$x$}, ylabel={$y$}, axis equal,width=\textwidth }}
%	\pgfplotsset{every x tick label/.append style={font=\small, yshift=0.5ex}}
%	\pgfplotsset{every y tick label/.append style={font=\small, xshift=0.5ex}}
	\renewcommand{\d}[1]{\ensuremath{\operatorname{d}\!{#1}}}
	\newcommand{\pydx}[2]{\frac{\partial #1}{\newcommand\partial #2}}
	\newcommand{\dydx}[2]{\frac{\d #1}{\d #2}}
	\newcommand{\ddx}[1]{\frac{\d{}}{\d{#1}}}
	\newcommand{\evat}[3]{\left. #1\right|_{#2}^{#3}}
	\newcommand{\restr}[2]{\evat{#1}{#2}{}}
\begin{document} 

% These commands set up the running header on the top of the exam pages
\pagestyle{head}
\firstpageheader{}{}{}
\runningheader{\class}{\examnum\ - Page \thepage\ of \numpages}{\examdate}
\runningheadrule

\begin{flushright}
\begin{tabular}{p{2.8in} r l}
\textbf{\class} & \textbf{Name (Print):} & \makebox[2in]{\underline{David DeMark (KEY)}}\\
\textbf{\term} &&\\
\textbf{\examnum} &&\\
\textbf{\examdate} &&\\
\textbf{Time Limit: \timelimit} & Section & \makebox[2in]{\hrulefill}
\end{tabular}\\
\end{flushright}
\rule[1ex]{\textwidth}{.1pt}

You may \textit{not} use your books, notes, graphing calculator, phones or any other internet devices on this exam. Please show all work clearly and legibly.\\

%You are \textbf{required} to justify your answers rigorously on each problem on this quiz. Supporting evidence and/or informal justification may be redeemed for partial credit.\\
\hspace*{12cm}\begin{minipage}[t]{2.3in}
\vspace{0pt}
%\cellwidth{3em}
\gradetablestretch{2}
\vqword{Problem}
\addpoints % required here by exam.cls, even though questions haven't started yet.	
\gradetable[v]%[pages]  % Use [pages] to have grading table by page instead of question

\end{minipage}
%\newpage % End of cover page

%%%%%%%%%%%%%%%%%%%%%%%%%%%%%%%%%%%%%%%%%%%%%%%%%%%%%%%%%%%%%%%%%%%%%%%%%%%%%%%%%%%%%
%
% See http://www-math.mit.edu/~psh/#ExamCls for full documentation, but the questions
% below give an idea of how to write questions [with parts] and have the points
% tracked automatically on the cover page.
%
%
%%%%%%%%%%%%%%%%%%%%%%%%%%%%%%%%%%%%%%%%%%%%%%%%%%%%%%%%%%%%%%%%%%%%%%%%%%%%%%%%%%%%%


\begin{questions}

% Basic question
%\vspace*{-130pt}
\addpoints
%\begin{changemargin}{0pt}{137.4803pt} Find the real numbers within the interval $I$ at which $f$ is \emph{not} continuous. State whether $f$ is continuous from the right, the left, or neither.
%\begin{parts}
%\part[1] $\displaystyle{f(x)=\frac{1}{x^2+4x-21}}$\hfill $I=(-\infty,\infty)$\vskip1.5mm
%\part[1] $\displaystyle{f(x)=\frac{\sin{\frac{x}{2}}}{\cos{2x}}}$\hfill $I=[-\pi,\pi]$\vskip1.5mm
%\part[1] $\displaystyle{f(x)=\begin{cases}2^x&x\leq 3\\-3x+17&3<x\leq 4\\\frac{1}{\sqrt{x-3}}&x>4
%	\end{cases}}$\hfill $I=(-3,8)$
%\end{parts}
%\end{changemargin}
\question
\vspace*{-130pt}
\begin{changemargin}{0pt}{137.4803pt}
\begin{parts}
\part[4] State the mean value theorem.\end{parts}\begin{proof}[Response] For real numbers $a<b$, \textbf{if}\footnote{Note the \textbf{If... Then...} structure to my response. The \textbf{if} is the hypothesis and the \textbf{then} is the conclusion. Both parts are necessary to any theorem!!!} $f$ is a function which is continuous on $[a,b]$ and differentiable on $(a,b)$, \textbf{then} there is a real number $c$ with $a<c<b$ such that $f'(c)=\frac{f(a)-f(b)}{a-b}$. \end{proof}\end{changemargin}\vspace{5em}\begin{parts}\setcounter{partno}{1}
\part[3]For $f(x)=x^3+x+4$  find all c in the interval $(-2,0)$ which satisfy the statement of the mean value theorem for $f(x)$ on the interval $[-2,0]$.\end{parts}
\begin{proof}[Response]
	We compute that $f(0)=4$ and $f(-2)=-6$. Thus, we are looking for $c$ such that $f'(c)=\frac{f(0)-f(-2)}{0-(-2)}=\frac{4-(-6)}{2}=5$. We set $f'(c)=3c^2+1=5$. Solving that yields $c=\pm \sqrt{\frac{4}{3}}=\pm\frac{2}{\sqrt{3}}$. As only $-\frac{2}{\sqrt{3}}$ is in our interval $(-2,0)$, that is our only such $c$ value.
\end{proof}

\newpage
\addpoints
\question[6] Let $g(x)=x^5-10x$. Find where $g$ is increasing and decreasing and find any local maxima or minima.
\begin{proof}[Response]
	As we will refer to both, we compute $g'(x)$ and $g''(x)$ upfront:
	\begin{align*}
		g'(x)&=5x^4-10=5(x^4-2)\\
		g''(x)&=20x^3
	\end{align*}
	To find intervals of increase \& decrease, we need to determine when $f'(x)>0$. To do that, we solve for critical points. $g$ is differentiable everywhere, so all critical points are of the type $g'(x)=0$. We set $g'(x)=x^4-2=(x^2-\sqrt{2})(x^2+\sqrt{2})=(x-\sqrt[4]{2})(x+\sqrt[4]{2})(x^2+\sqrt{2})=0$ and get $x=\pm \sqrt[4]{2}$. Thus, our intervals of interest are $(-\infty,-\sqrt[4]{2})$, $(-\sqrt[4]{2},\sqrt[4]{2})$ and $(\sqrt[4]{2},\infty)$. 
	
	Plugging in $x=-2$ gives $g'(-2)=5(2^4-2)=70$. Plugging in $x=0$ gives $g'(0)=-10$. Finally, by even symmetry, plugging in $x=2$ gives $g'(2)=70$. Thus, $g$ is increasing on $(-\infty,-\sqrt[4]{2})\cup(\sqrt[4]{2},\infty)$ and decreasing on $(-\sqrt[4]{2},\sqrt[4]{2})$. We note $g''(x)$ is positive when $x>0$ and negative when $x<0$. Thus, $x=-\sqrt[4]{2}$ is a local maximum and $x=\sqrt[4]{2}$ is a local minimum by the second derivative test.
\end{proof}\vspace{1em}
\question Use L'hopital's rule to compute the following limits:
\begin{parts}
\part[3] \[\lim_{x\to 0}\frac{e^x-1}{x}=\]\begin{proof}[Response]
Note $\displaystyle\lim_{x\to 0}e^x-1=0=\displaystyle\lim_{x\to 0}x$. Thus, 	\begin{align*}\lim_{x\to 0}\frac{e^x-1}{x}&=\lim_{x\to 0}\frac{\ddx{x}(e^x-1)}{\ddx{x}(x)}=\lim_{x\to 0}\frac{e^x}{1}=\frac{e^0}{1}=1\end{align*}
\end{proof}
\part[4] \[\lim_{x\to 0}\frac{\sin(3x)}{\tan(7x)}=\]\begin{proof}[Response]
	Note $\displaystyle\lim_{x\to 0}\sin(3x)=0=\displaystyle\lim_{x\to 0}\tan(7x)$. Thus, 	\begin{align*}\lim_{x\to 0}\frac{\sin(3x)}{\tan(7x)}=\lim_{x\to 0}\frac{\ddx{x}\sin(3x)}{\ddx{x}\tan(7x)}=\lim_{x\to 0}\frac{3\cos(3x)}{7\sec^2(7x)}=\frac{3}{7}\lim_{x\to 0}\frac{\cos(3x)}{\sec^2(7x)}=\frac{3}{7}*\frac{1}{1}=\frac{3}{7}\end{align*}

\end{proof}
\end{parts}
\addpoints

\end{questions}
\end{document}